%%%%%%%%%%%%%%%%%%%%%%%%%%%%%%%%%%%%%%%%%
% Beamer Presentation
% LaTeX Template
% Version 1.0 (10/11/12)
%
% This template has been downloaded from:
% http://www.LaTeXTemplates.com
%
% License:
% CC BY-NC-SA 3.0 (http://creativecommons.org/licenses/by-nc-sa/3.0/)
%
%%%%%%%%%%%%%%%%%%%%%%%%%%%%%%%%%%%%%%%%%

%----------------------------------------------------------------------------------------
%	PACKAGES AND THEMES
%----------------------------------------------------------------------------------------

\documentclass{beamer}

\mode<presentation> {
	
	% The Beamer class comes with a number of default slide themes
	% which change the colors and layouts of slides. Below this is a list
	% of all the themes, uncomment each in turn to see what they look like.
	
	%\usetheme{default}
	%\usetheme{AnnArbor}
	%\usetheme{Antibes}
	%\usetheme{Bergen}
	%\usetheme{Berkeley}
	%\usetheme{Berlin}
	%\usetheme{Boadilla}
	%\usetheme{CambridgeUS}
	%\usetheme{Copenhagen}
	%\usetheme{Darmstadt}
	%\usetheme{Dresden}
	%\usetheme{Frankfurt}
	%\usetheme{Goettingen}
	%\usetheme{Hannover}
	%\usetheme{Ilmenau}
	%\usetheme{JuanLesPins}
	%\usetheme{Luebeck}
	\usetheme{Madrid}
	%\usetheme{Malmoe}
	%\usetheme{Marburg}
	%\usetheme{Montpellier}
	%\usetheme{PaloAlto}
	%\usetheme{Pittsburgh}
	%\usetheme{Rochester}
	%\usetheme{Singapore}
	%\usetheme{Szeged}
	%\usetheme{Warsaw}
	
	% As well as themes, the Beamer class has a number of color themes
	% for any slide theme. Uncomment each of these in turn to see how it
	% changes the colors of your current slide theme.
	
	%\usecolortheme{albatross}
	%\usecolortheme{beaver}
	%\usecolortheme{beetle}
	%\usecolortheme{crane}
	%\usecolortheme{dolphin}
	%\usecolortheme{dove}
	%\usecolortheme{fly}
	%\usecolortheme{lily}
	%\usecolortheme{orchid}
	%\usecolortheme{rose}
	%\usecolortheme{seagull}
	%\usecolortheme{seahorse}
	%\usecolortheme{whale}
	%\usecolortheme{wolverine}
	
	%\setbeamertemplate{footline} % To remove the footer line in all slides uncomment this line
	%\setbeamertemplate{footline}[page number] % To replace the footer line in all slides with a simple slide count uncomment this line
	
	%\setbeamertemplate{navigation symbols}{} % To remove the navigation symbols from the bottom of all slides uncomment this line
}

\usepackage{graphicx} % Allows including images
\usepackage{booktabs} % Allows the use of \toprule, \midrule and \bottomrule in tables

%----------------------------------------------------------------------------------------
%	TITLE PAGE
%----------------------------------------------------------------------------------------

\title[SURE-LET APPROACH]{Main Topic } % The short title appears at the bottom of every slide, the full title is only on the title page

\author{Hoang Duc Viet} % Your name
\institute[USTH] % Your institution as it will appear on the bottom of every slide, may be shorthand to save space
{
	ICT Lab \\ % Your institution for the title page
	\medskip
	\textit{viethdweb@gmail.com} % Your email address
}
\date{\today} % Date, can be changed to a custom date

\begin{document}
	
	\begin{frame}
		\titlepage % Print the title page as the first slide
	\end{frame}
	
	\begin{frame}
		\frametitle{Overview} % Table of contents slide, comment this block out to remove it
		\tableofcontents % Throughout your presentation, if you choose to use \section{} and \subsection{} commands, these will automatically be printed on this slide as an overview of your presentation
	\end{frame}
	
	%----------------------------------------------------------------------------------------
	%	PRESENTATION SLIDES
	%----------------------------------------------------------------------------------------
	
	%------------------------------------------------
	\section{INTRODUCTION} % Sections can be created in order to organize your presentation into discrete blocks, all sections and subsections are automatically printed in the table of contents as an overview of the talk
	%------------------------------------------------
	
	\subsection{} % A subsection can be created just before a set of slides with a common theme to further break down your presentation into chunks
	
	\begin{frame}
		\frametitle{INTRODUCTION}
		\begin{itemize}
	\item A new approach to image denoising,based on the image-domain minimization of an estimate of the
		mean squared error:
		
	\textbf{Stein's unbiased risk estimate (SURE)}
		\item The denoising process can
		be expressed as a linear combination of elementary denoising
		processes: 
		
	\textbf{Linear expansion of thresholds (LET)}
		\item  Evaluate this denoising performances by comparing PSNR
	\end{itemize}
	
\textbf{$\Rightarrow$ SURE-LET Approach}
\end{frame}
	
	%------------------------------------------------
	
	\begin{frame}
		\frametitle{Topic Points}
		\begin{itemize}
			\item  Theoretical Background
		
		\
		
			\item SURE-LET Formula 
			
			\
			
			\item Summary of the Algorithm

		
		\end{itemize}
	\end{frame}
	
	%------------------------------------------------
	
	\begin{frame}
		\frametitle{Theoretical Background}
\begin{block}{MSE}

$MSE=\frac{1}{N}\displaystyle\sum_{n=1}^{N}|\hat{x_n}-x_n|^2 \Leftrightarrow MSE=\frac{1}{N}||\hat{x}-x||^2$ 
	\end{block}


	\end{frame}
	
	%------------------------------------------------
	\begin{frame}
			\frametitle{Theoretical Background (1)}
	\begin{block}{Lemma}
$\varepsilon\left\{\displaystyle\sum_{n=1}^{N}f_n(y)x_n\right\}=\varepsilon\left\{\displaystyle\sum_{n=1}^{N}f_n(y)y_n\right\}-
\sigma^2\varepsilon\underbrace{\left\{\displaystyle\sum_{n=1}^{N}\frac{\partial f_n(y)}{\partial y_n}\right\}}_{div{F(y)}}$
	\end{block}

\end{frame}
		%------------------------------------------------
\begin{frame}
	\frametitle{Theoretical Background (2)}
	\begin{block}{Theorem}
	Theorem 1: Under the same hypotheses as Lemma 1, the
	random variable
	
	\
	
	$\epsilon=\frac{1}{N}||F(y)-y||^2 + \frac{2\sigma^2}{N}div\{F(y)\}-\sigma^2$

\

is an unbiased estimator of the MSE :	

\

$\varepsilon\{\epsilon\}=\frac{1}{N}\varepsilon \left\{||F(y)-x||^2\right\}$


\end{block}
\end{frame}	
		%------------------------------------------------
\begin{frame}
	\frametitle{SURE-LET Formula }
\begin{block}{LET}
(LET: linear expansion of thresholds)

$F(y)=\displaystyle\sum_{k=1}^{K}a_kF_k(y) $
\end{block}


\end{frame}	
	%------------------------------------------------
	\begin{frame}
		\frametitle{SURE-LET Formula}
		From Theorem and Let above, we combine them so we have SURE-LET formula as below : 	
		
		\

$\displaystyle\sum_{l=1}^{K}\underbrace{F_k(y)^T F_l(y)a_l}_{[M]_{k,l}} = \underbrace{F_k(y)^Ty-\sigma^2div\left\{F_k(y)\right\}}_{[c]_k}$    $$for \ k = 1,2...K$$	
\begin{center}
$\Updownarrow$ \ \ \ \ \ \ \ \ \ \ \ \ \ \ \ \ \ \ \ \ \

\

\textbf{Ma=c} \ \ \ \ \ \ \ \ \ \ \ \ \ \ \ \ \ \ \ \
\end{center}






\end{frame}	
	
	
	
	
	
	%------------------------------------------------
	\begin{frame}
		\frametitle{ Pointwise SURE-LET Transform Denoising}
		A function of the noisy
		input coefficients :
		\begin{center}
			$\hat{x}=F(y)=R\Theta(Dy)$
		\end{center}
		
		A linear
		expansion of denoising algorithms $F_k$ :
		\begin{center}
			$F(y)=\displaystyle\sum_{k=1}^{K}a_k \ \underbrace{R\Theta_k(Dy)}_{F_{k(y)}}$
			
			\
			
where \ $\Theta_k(.)$ \ are elementary pointwise thresholding functions.
		\end{center}
		
	\end{frame}
	
	%------------------------------------------------
	
	%------------------------------------------------
	
	\begin{frame}
		\frametitle{Pointwise SURE-LET Transform Denoising 1}
\begin{itemize}
	\item Evaluation of the Divergence Term ----- $\alpha$ 
	\item Influence of the Boundary Extensions
	\item Applications to Standard Linear Transforms
\end{itemize}

\end{frame}
	
	%------------------------------------------------
	
	\begin{frame}
		\frametitle{Pointwise SURE-LET Transform Denoising 2}

	
	\
	\begin{block}{\textbf{Corollary:}}
$\epsilon=\frac{1}{N}||F(y)-y||^2+\frac{2\sigma^2}{N}\alpha^T\Theta'(Dy)-\sigma^2$

\	

where
\begin{itemize}
	\item $\alpha=diag \{DR\}=\left\{[DR]_{1,1},[DR]_{2,2},..........,[DR]_{L,L}\right\}$ \ is
	a vector made of the diagonal elements of the matrix DR;
\item $\Theta'(DY)=\Theta'(w)=(\theta'_i(w_i))_{i\in[1;L]}$
\end{itemize}

\end{block}
	
	

	
	\end{frame}
	
	%------------------------------------------------
	
	\begin{frame}
\frametitle{Pointwise SURE-LET Transform Denoising 3}
	\textbf{Evaluation of the Divergence Term ----- $\alpha$ :}
	
	\
	
	
	An approximate value $\hat{\alpha}$ for $diag \{DR\}$ in finally obtained by averaging the realizations $v_i$ over I runs(typically, I = 1000 provides great accuracy)
	
	\begin{center}
		$\hat\alpha=\frac{1}{I}\displaystyle\sum_{i=1}^{I}v_i$
	\end{center}
	\end{frame}
	
	%------------------------------------------------
	
	\begin{frame}
		\frametitle{Pointwise SURE-LET Transform Denoising 4}
Lemma 2: Let \textbf{b} be a normalized Gaussian white noise with \textbf{L} components. Then, we have the following equality:

\

$\varepsilon\left\{diag \{DRbb^T\}\right\}=diag\left\{DR\right\}$	

\

Proof:

\

$\varepsilon\left\{diag \{DRbb^T\}\right\}=diag\left\{DR\underbrace{\varepsilon \{bb^T\}}_{Id}\right\}$
$$=diag\left\{DR\right\}$$
	\end{frame}
	
	%------------------------------------------------
	
	\begin{frame}{Pointwise SURE-LET Transform Denoising 5}
\textbf{Influence of the Boundary Extensions:}
	
	\
	
	Usual boundary extensions are linear preprocessing applied to the available data \textbf{y} and can, therefore, be expressed in a matrix form. In particular, for a given boundary extension of length \textbf{E}, i.e., characterized by an ExN matrix \textbf{H}, the denoising process becomes
	
	\

$F(y)=[Id_N \ 0_{N \times E}]R_{N+E^{\Theta}}\left(D_{N+E}\left[\begin{tabular}{c}
y \\ 
Hy \\ 
\end{tabular} \right] \right)$
	
	\
	
\ \ \ \ \ \ \ $=R'\Theta(D'y)$
	

	

\end{frame}

	
	
	%------------------------------------------------
	

		\begin{frame}{Pointwise SURE-LET Transform Denoising 6}

\textbf{Applications to Standard Linear Transforms:}
\begin{block}{Nonredundant transforms:}
When is nonredundant, the divergence term $\alpha$ in Corollary is given by

\

$\alpha= \underbrace{ \left[ 1,1,....,1\right]^T}_{L times}$
\end{block}




	
	
	\end{frame}
	

	
	%------------------------------------------------
	
	
	
	\begin{frame}{Pointwise SURE-LET Transform Denoising 7}
	
	\begin{block}{Undecimated filterbank transforms:}
		Lemma 4: When \textbf{D} and \textbf{R} are periodically extended implementations of the analysis-synthesis filterbank, the divergence
		term in Corollary is given by $\alpha=[{\alpha_1}^T, {\alpha_2}^T,....,{\alpha_J}^T]^T $ where
		
		\
		
		\begin{center}
				$\alpha_i= \left(\displaystyle\sum_{n}\gamma_i[nN]\right)\underbrace{ \left[ 1,1,....,1\right]^T}_{N times}$
				
				\
				
				and where $\gamma_i[n]$ is the $n$th coefficient of the $\tilde{G_i}(z^{-1})G_i(z)$ filter .
			
		\end{center}
	
	\end{block}
	
\end{frame}

\begin{frame}{Summary of the Algorithm}
\begin{itemize}
	\item Perform a boundary extension on the noisy image.
	\item Perform an UWT on the extended noisy image.
	\item \textbf{For} $i=1...J$ (number of bandpass subbands), \textbf{For} $k=1,2:$
	\begin{itemize}
		\item Apply the pointwise thresholding functions $t_k$ defined
		in \textbf{A. Choosing an Efficient Thresholding Function} to the current subband $w_i$.
\item Reconstruct the processed subband by setting all the
other subbands to zero to obtain $F_{i,k}(y)$	
\item Compute the first derivative of $t_k$ for each coefficient
of the current subband $w_i$ and build the corresponding
coordinate of \textbf{c} as exemplified by \textbf{SURE-LET Formula}
\end{itemize}
	
	\item Compute the matrix \textbf{M} and deduce the optimal-in the
	minimum SURE sense-linear parameters ’s using the
	matrix formulation of \textbf{SURE-LET Formula}
	\item  The noise-free image $\hat{x}$ is finally estimated by the sum of
	each $F_{i,k}$ weighted by its corresponding SURE-optimized $a_{i,k}$
\end{itemize}
\end{frame}
%------------------------------------------------
\begin{frame}{Summary of the Algorithm 1}
\begin{block}{Choosing an efficient thresholding function}
A good choice has been experimentally found to be of the form.
\begin{center}
	$\theta(w)=a_i,1{t_1}(w)+a_i,2{t_2}(w)$
	
\end{center}
\end{block}

\begin{block}{Where}
	\begin{center}
		$t_1(w)=w \& t_2(w)=w\left(1-e^{-(\frac{\omega}{3\sigma})}^8 \right)$
	\end{center}
\end{block}
	
\end{frame}
	



%------------------------------------------------
\section{COMPARISON \& RESULT}	
	
	
	\begin{frame}{COMPARISON \& RESULTS}
	\begin{itemize}
		\item  Wavelet-Domain Versus Image-Domain Optimization
		\item Periodic Versus Symmetric Boundary Extensions
		\item Comparison With State-of-the-Art Denoising Schemes
	\end{itemize}
	
	
\end{frame}



%------------------------------------------------
	\begin{frame}{COMPARISON \& RESULTS 1}
	\begin{center}
		\includegraphics{mse1.png}
	\end{center}
\end{frame}

%------------------------------------------------
	
\begin{frame}{COMPARISON \& RESULTS 2}
\begin{center}
	\includegraphics{result4.png}
\end{center}

\end{frame}


%------------------------------------------------
	
	
	\begin{frame}{COMPARISON \& RESULTS 3}
	
		(a) Part of the noise-free Boat image. (b) A noisy version of it: PSNR =
	22:11 dB. (c) BiShrink denoising result: PSNR = 29:99 dB. (d) ProbShrink
	denoising result: PSNR = 29:97 dB. (e) BLS-GSM denoising result: PSNR =
	30:36 dB. (f) UWT SURE-LET denoising result: PSNR = 30:24 dB.
	
\end{frame}	
%------------------------------------------------
\begin{frame}{CONCLUSION}
SURE-LET denoising procedure gives quite a decent visual quality compared to the best state-of-the-art spatially adaptive method.
\end{frame}

	
	
	
	
	
	
	
	
	
	
	
	
	
	
	
	
	
	
	
	
	
	
	
	
	
	
	
	
	
	\begin{frame}
		\Huge{\centerline{The End}}
	\end{frame}
	
	%----------------------------------------------------------------------------------------
	
\end{document}